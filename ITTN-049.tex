\documentclass[PMO,authoryear,toc]{lsstdoc}
% lsstdoc documentation: https://lsst-texmf.lsst.io/lsstdoc.html
\input{meta}

% Package imports go here.

% Local commands go here.

%If you want glossaries
%\input{aglossary.tex}
%\makeglossaries

\title{Internet Edge Firewall Design}

% Optional subtitle
% \setDocSubtitle{A subtitle}

\author{%
Julio Constanzo
}

\setDocRef{ITTN-049}
\setDocUpstreamLocation{\url{https://github.com/lsst-it/ittn-049}}

\date{\vcsDate}

% Optional: name of the document's curator
% \setDocCurator{The Curator of this Document}

\setDocAbstract{%
This document describe the internet edge firewall design, actual setup migration and integration with Rubin's monitoring system.
}

% Change history defined here.
% Order: oldest first.
% Fields: VERSION, DATE, DESCRIPTION, OWNER NAME.
% See LPM-51 for version number policy.
\setDocChangeRecord{%
  \addtohist{1}{YYYY-MM-DD}{Unreleased.}{Julio Constanzo}
}


\begin{document}

% Create the title page.
\maketitle
% Frequently for a technote we do not want a title page  uncomment this to remove the title page and changelog.
% use \mkshorttitle to remove the extra pages

% ADD CONTENT HERE
% You can also use the \input command to include several content files.

\appendix
% Include all the relevant bib files.
% https://lsst-texmf.lsst.io/lsstdoc.html#bibliographies
\section{References} \label{sec:bib}
\renewcommand{\refname}{} % Suppress default Bibliography section
\bibliography{local,lsst,lsst-dm,refs_ads,refs,books}

% Make sure lsst-texmf/bin/generateAcronyms.py is in your path
\section{Acronyms} \label{sec:acronyms}
\addtocounter{table}{-1}
\begin{longtable}{p{0.145\textwidth}p{0.8\textwidth}}\hline
\textbf{Acronym} & \textbf{Description}  \\\hline

AC & Alternating Current \\\hline
AD & Associate Director \\\hline
AOC &  AURA Oversight Council \\\hline
AURA & Association of Universities for Research in Astronomy \\\hline
CE & Communications Engagement \\\hline
CPU & Central Processing Unit \\\hline
DE & dark energy \\\hline
DNS & Domain Name Service \\\hline
DOM & Document Object Model \\\hline
FS & File System \\\hline
FY21 & Financial Year 21 \\\hline
GB & Gigabyte \\\hline
GUI & Graphical User Interface \\\hline
HDD &  Hard Disk Drive \\\hline
HTTP & HyperText Transfer Protocol \\\hline
IP & Internet Protocol \\\hline
IPS & Integrated Project Schedule \\\hline
IPsec & Internet Protocol Security \\\hline
IT & Information Technology \\\hline
LAN & Local Area Network \\\hline
LED & Light-Emitting Diode \\\hline
NAT & Network Address Translation \\\hline
PCI & Peripheral Component Interconnect \\\hline
PMO & Project Management Office \\\hline
PS & Project Scientist \\\hline
RAM & Random Access Memory \\\hline
SATA & Serial Advanced Technology Attachment \\\hline
SSD & Solid-State Disk \\\hline
TBD & To Be Defined (Determined) \\\hline
URL & Universal Resource Locator \\\hline
USB & Universal Serial Bus \\\hline
VLAN &  Virtual Local Area Network \\\hline
VPC &  Virtual Private Cloud \\\hline
VPN & virtual private network \\\hline
WAN & Wide Area Network \\\hline
\end{longtable}

% If you want glossary uncomment below -- comment out the two lines above
%\printglossaries





\end{document}
